

\section*{Description}

Excellentea is an automatic tea maker. The user can operate the machine remotely through an online interface. In other words, you can order your tea, by chosing from several modes, as you leave work and find it ready when you get home. It also comes with the option of controlling the machine using control buttons and an L\+CD display.

\section*{Usage}

\subsection*{Installation}

Clone the repository to your Raspberry Pi and run the following commands\+:


\begin{DoxyCode}
./configure
make
make install
\end{DoxyCode}
 The program can then be started by running the command


\begin{DoxyCode}
excellentea
\end{DoxyCode}
 from the terminal.

\subsection*{User operation}


\begin{DoxyEnumerate}
\item Load your cup with water
\end{DoxyEnumerate}

\subsection*{2. Load your tea infuser with the tea of your choice }

For remote control\+:


\begin{DoxyEnumerate}
\item Activate the tea maker from the online user interface
\end{DoxyEnumerate}

\subsection*{4. Choose the brewing mode of your tea }

For on-\/machine control\+:


\begin{DoxyEnumerate}
\item Activate using inbuilt interface
\end{DoxyEnumerate}

\subsection*{4. Navigate with control buttons and select your preferred program }


\begin{DoxyEnumerate}
\item Wait...tea is ready\+:)
\end{DoxyEnumerate}

\section*{Hardware}

\subsection*{Key components}


\begin{DoxyItemize}
\item 1 Raspberry PI microcontroller board (tested on version 3 Model B)
\item 1 Stepper motor (M\+I\+K\+R\+O\+E-\/1530)
\item 1 Digital temperature sensor (\mbox{\hyperlink{classds18b20}{ds18b20}})
\item 12V DC power supply
\item 1 heating element (12V) (B004\+O8\+B\+G\+XE)
\item 1 tea infuser
\item 1 reed float sensor (59630)
\item 2 18-\/pin through hole socket (E\+D18\+DT)
\item 2 Darlington transistor array (U\+L\+N2803A)
\item L\+CD
\item 2 N-\/channel logic-\/level M\+O\+S\+F\+ET (F\+Q\+P30\+N06L)
\item M\+O\+S\+F\+ET heat sink (507222\+B00000G)
\end{DoxyItemize}

\subsection*{Additional components}

The project also requires standard passive components (e.\+g. resistors), prototyping tools (e.\+g. breadboard/pcb) and materials for the encasing. See the \href{Main.sch}{\tt circuit schematics} for details.

\subsection*{Protocol}

The digital temperature sensor D\+S18\+B20 communicates with the board through a 1-\/wire protocol on pin 7 (B\+C\+M4). The reed float sensor only outputs two-\/states so a communication protocol is not required.

\subsection*{Prerequisites}

The raspberry PI must be connected to the internet for remote access.

\section*{Software}

\subsection*{Flow diagram}



\section*{Authors}


\begin{DoxyItemize}
\item \href{https://github.com/andreaspanou}{\tt {\bfseries Andrea Spanou}} -\/ {\itshape Initial work}
\item \href{https://github.com/CiaranAnthony}{\tt {\bfseries Ciaran Mc\+Geady}} -\/ {\itshape Initial work}
\item \href{https://github.com/SimoneMarcigaglia}{\tt {\bfseries Simone Marcigaglia}} -\/ {\itshape Initial work}
\end{DoxyItemize}

See also the list of \href{https://github.com/GlasgowTeam3RTEP/ExcellenTea/contributors}{\tt contributors} who participated in this project.

\section*{Contributing}

Please read \mbox{\hyperlink{md_CONTRIBUTING}{C\+O\+N\+T\+R\+I\+B\+U\+T\+I\+NG.md}} for details on our code of conduct, and the process for submitting pull requests to us.

\section*{License}

This project is licensed under the M\+IT License -\/ see the \mbox{[}L\+I\+C\+E\+N\+SE\mbox{]}(L\+I\+C\+E\+N\+SE) file for details.

\section*{Acknowledgments}

We would like to thank the weather in Glasgow for making us think about tea all the time. 